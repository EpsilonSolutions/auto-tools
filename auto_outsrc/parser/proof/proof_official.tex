%%%%%%%%%%%%%%%%%%%%%%%%%%%%%%%%%%%%%%%%%%
% STATIC LATEX CONTENT FOR MACHINE-GENERATED PROOFS
%
% THIS FILE IS NOT GENERATED. IT INCLUDES THE FILE "proof_schemename.tex"
% WHICH CONTAINS THE MACROS FOR THE GUTS OF THE PROOF.
%
% YOU SHOULDN'T NEED TO EDIT THIS FILE EXCEPT FOR STYLISTIC
% REASONS, I.E., YOU DON'T NEED TO CUSTOMIZE IT FOR A GIVEN SCHEME
%%%%%%%%%%%%%%%%%%%%%%%%%%%%%%%%%%%%%%%%%%

\documentclass[11pt]{article}
\usepackage{fullpage,amsthm,amsmath}
\usepackage{latexsym,amssymb,xspace}

\begin{document}
% input for automatically generated outsource proof


\catcode`\^ = 13 \def^#1{\sp{#1}{}}
\newcommand{\newln}{\\&\quad\quad{}}
\newcommand{\schemename}{{\sf Dse09}}
\newcommand{\schemeref}{Dse09_proof}
\newcommand{\schemecite}{\cite{REF}}
\newcommand{\randomsecretkeys}{ RD_1,RD_2,RD_3,RD_4,RD_5,RD_6,RD_7,RK,Rtagk }
\newcommand{\secretkey}{ D_1,D_2,D_3,D_4,D_5,D_6,D_7,K,tagk }
\newcommand{\listofbfs}{ {\sf uf_0},{\sf bf_0},{\sf uf_1} }
\newcommand{\listofmskvalues}{ \alpha,a_1 }
\newcommand{\listofrandomvalues}{ r_1,r_2,z_2,z_1,idhash,tagk }
\newcommand{\keydefinitions}{ D_1' = (D_1 ) ^ { {\sf uf_0}},\alpha' = {\alpha \cdot {\sf bf_0}},z_1' = {z_1 \cdot {\sf bf_0}},r_2' = {r_2 \cdot {\sf bf_0}},r_1' = {r_1 \cdot {\sf bf_0}},z_2' = {z_2 \cdot {\sf bf_0}},K' = (K ) ^ { {\sf uf_1}},tagk' = {tagk \cdot {\sf bf_0}} }
\newcommand{\originalkey}{ D_1 = g_2^{\alpha \cdot a_1} \cdot v_2^{(r_1 + r_2)},D_2 = g_2^{-\alpha} \cdot v1_2^{(r_1 + r_2)} \cdot g_2^{z_1},D_3 = g_2^{b \cdot -z_1},D_4 = v2_2^{(r_1 + r_2)} \cdot g_2^{z_2},D_5 = g_2^{b \cdot -z_2},D_6 = g_2^{b \cdot r_2},\\ D_7 = g_2^{r_1},K = (u_2^{idhash} \cdot w_2^{tagk} \cdot h_2)^{r_1},tagk = tagk }
\newcommand{\transformkey}{ {D_1'} = {D_1}^{{{\sf uf_0}}},{D_2'} = {D_2}^{{{\sf bf_0}}},{D_3'} = {D_3}^{{{\sf bf_0}}},{D_4'} = {D_4}^{{{\sf bf_0}}},{D_5'} = {D_5}^{{{\sf bf_0}}},{D_6'} = {D_6}^{{{\sf bf_0}}}, \\ {D_7'} = {D_7}^{{{\sf bf_0}}},K' = K^{{{\sf uf_1}}},tagk' = tagk^{{{\sf bf_0}}} }
\newcommand{\pseudotransformkey}{ {\bar{D_1}} = {RD_1}^{{{\sf uf_0}}},{\bar{D_2}} = {RD_2}^{{{\sf bf_0}}},{\bar{D_3}} = {RD_3}^{{{\sf bf_0}}},{\bar{D_4}} = {RD_4}^{{{\sf bf_0}}},{\bar{D_5}} = {RD_5}^{{{\sf bf_0}}},{\bar{D_6}} = {RD_6}^{{{\sf bf_0}}}, \\ {\bar{D_7}} = {RD_7}^{{{\sf bf_0}}},\bar{K} = RK^{{{\sf uf_1}}},\bar{tagk} = Rtagk^{{{\sf bf_0}}} }
\newcommand{\expandedtransformkey}{ D_1' = (g_2^{\alpha \cdot a_1} \cdot v_2^{(r_1 + r_2)})^{{{\sf uf_0}}},D_2' = g_2^{-\alpha'} \cdot v1_2^{(r_1' + r_2')} \cdot g_2^{z_1'},D_3' = g_2^{b \cdot -z_1'},D_4' = v2_2^{(r_1' + r_2')} \cdot g_2^{z_2'},D_5' = g_2^{b \cdot -z_2'},D_6' = g_2^{b \cdot r_2'}, \\ D_7' = g_2^{r_1'},K' = u_2^{idhash} \cdot w_2^{tagk} \cdot h_2^{r_1 \cdot {{\sf uf_1}}},tagk' = tagk' }


\newtheorem{definition}{Definition}
\newtheorem{theorem}{Theorem}
\newcommand{\Oracle}{\mathcal{O}}
\newcommand{\Adv}{\mathcal{A}}
\newcommand{\Bdv}{\mathcal{B}}
\newcommand{\MS}{\mathcal{M}}
\newcommand{\Psetup}{\mathsf{PSetup}}
\newcommand{\Msetup}{\mathsf{Setup}}
\newcommand{\params}{\mathit{params}}
\newcommand{\brk}[1]{\langle #1 \rangle}
\newcommand{\ait}[1]{#1}
\newcommand{\Ga}{\ait{\mathbb{G}}_1}
\newcommand{\ga}{\ait{g}_1}
\newcommand{\ha}{\ait{h}_1}
\newcommand{\poly}{\mathrm{poly}}

\newcommand{\bit}[1]{#1}
\newcommand{\Gb}{\bit{\mathbb{G}}_2}
\newcommand{\gb}{\bit{g}_2}
\newcommand{\hb}{\bit{h}_2}

\newcommand{\cit}[1]{#1}
\newcommand{\Gc}{\cit{\Group_T}}
\newcommand{\gc}{\cit{g}}
\newcommand{\hc}{\cit{h}}
\newcommand{\Zp}{\mathbb{Z}_p}

\newcommand{\Group}{\ensuremath{\mathbb{G}}\xspace}
\newcommand{\Hroup}{\ensuremath{\mathbb{H}}\xspace}
\newcommand{\map}{\mathbf{e}}

\newcommand{\prot}{\mathsf{Prot}}
\newcommand{\auxext}{\mathit{auxext}}
\newcommand{\auxsim}{\mathit{auxsim}}
\newcommand{\aux}{\mathit{aux}}
\newcommand{\state}{\mathit{state}}
\newcommand{\Alg}{\mathsf{Alg}}
\newcommand{\A}{\mathcal{A}}

\newcommand{\Sig}{\mathsf{Sig}}
\newcommand{\G}{\mathsf{Gen}}
\newcommand{\TK}{\mathsf{TK}}
\newcommand{\pseudokey}{{\sf TK}_{sim}}
\newcommand{\SK}{\mathsf{SK}}
\newcommand{\PK}{\mathsf{PK}}
\newcommand{\MSK}{\mathsf{MSK}}
\newcommand{\CT}{\mathsf{CT}}
\newcommand{\Screen}{\mathsf{Screen}}
\newcommand{\Setup}{\mathsf{Setup}}
\newcommand{\Keygen}{\mathsf{Keygen}}
\newcommand{\KeygenOut}{\mathsf{KeygenOut}}
\newcommand{\Transform}{\mathsf{Transform}}
\newcommand{\Decrypt}{\mathsf{Decrypt}}
\newcommand{\Decout}{\mathsf{DecOut}}
\newcommand{\compareequals}{\stackrel{?}{=}}
\newcommand{\numsigs}{\eta}

\title{A machine-generated proof of security for {\schemename}}
\author{}
\date{}
\maketitle

\section{$\KeygenOut$ Proof}

Let $\listofmskvalues$ be the $\MSK$ variable(s) and $\listofrandomvalues$ be randomness selected in the $\Keygen$ algorithm. The $\Keygen$ algorithm runs to obtain the secret key, \\ $\SK = \{\secretkey\}$ and is computed as follows:

\begin{description}
\item {\sf SK}: \begin{multline*}  \originalkey \end{multline*}
\end{description}

\noindent
The $\KeygenOut$ algorithm selects blinding factors, $\listofbfs \in \Zp^*$. Let the new $\TK$ be computed as follows:

\begin{description}
\item {\sf $\TK$}: \begin{multline*}  \transformkey \end{multline*}
\end{description}

\noindent
Note that a simulator given only public parameters ($\PK$) can formulate a simulated $\pseudokey$ by randomly selecting $\randomsecretkeys$ and computing: 

\begin{description}
\item {\sf $\pseudokey$}: \begin{multline*}  \pseudotransformkey \end{multline*}
\end{description}

Observe that this simulated $\pseudokey$ has an identical distribution to $\TK$.  Let $\keydefinitions$ be the new $\MSK$ variables and random values selected in $\KeygenOut$, and $\TK$ is computed as follows:

\begin{description}
\item {\sf $\TK$}: \begin{multline*}  \expandedtransformkey \end{multline*} 
\end{description}

% TODO: fine-tune
%\section{$\Transform$ Proof}
%
%In the ${\schemename}$ scheme, we show the relevant portions to our transformation in the $\Decrypt$ algorithm. The ciphertext, $\CT = \{\ciphertext\}$ and the \\ $\SK = \{\secretkey\}$. 
%
%%\begin{description}
%%\item {Step 1: Compute tag} \begin{equation*} tag =  \end{equation*}
%%\end{description}
%\gutsofdecrypt
%
%% Rework this text...
%Then, we show how the computation of the original $\Decrypt$ algorithm is distributed between the $\Transform$ algorithm and the $\Decout$ algorithm.
%
%\gutsoftransform

\end{document}
