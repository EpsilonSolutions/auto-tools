%%%%%%%%%%%%%%%%%%%%%%%%%%%%%%%%%%%%%%%%%%
% STATIC LATEX CONTENT FOR MACHINE-GENERATED PROOFS
%
% THIS FILE IS NOT GENERATED. IT INCLUDES THE FILE "proof_schemename.tex"
% WHICH CONTAINS THE MACROS FOR THE GUTS OF THE PROOF.
%
% YOU SHOULDN'T NEED TO EDIT THIS FILE EXCEPT FOR STYLISTIC
% REASONS, I.E., YOU DON'T NEED TO CUSTOMIZE IT FOR A GIVEN SCHEME
%%%%%%%%%%%%%%%%%%%%%%%%%%%%%%%%%%%%%%%%%%

\documentclass[11pt]{article}
\usepackage{fullpage,amsthm,amsmath}
\usepackage{latexsym,amssymb,xspace}

\begin{document}
% input for automatically generated outsource proof
\input{proof_gendsewaters09}

\newtheorem{definition}{Definition}
\newtheorem{theorem}{Theorem}
\newcommand{\Oracle}{\mathcal{O}}
\newcommand{\Adv}{\mathcal{A}}
\newcommand{\Bdv}{\mathcal{B}}
\newcommand{\MS}{\mathcal{M}}
\newcommand{\Psetup}{\mathsf{PSetup}}
\newcommand{\Msetup}{\mathsf{Setup}}
\newcommand{\params}{\mathit{params}}
\newcommand{\brk}[1]{\langle #1 \rangle}
\newcommand{\ait}[1]{#1}
\newcommand{\G}{\ait{\mathbb{G}}}
\newcommand{\ga}{\ait{g}_1}
\newcommand{\ha}{\ait{h}_1}
\newcommand{\poly}{\mathrm{poly}}

\newcommand{\bit}[1]{#1}
%\newcommand{\Gb}{\bit{\mathbb{G}}_2}
\newcommand{\gb}{\bit{g}_2}
\newcommand{\hb}{\bit{h}_2}

\newcommand{\cit}[1]{#1}
\newcommand{\Gc}{\cit{\Group_T}}
\newcommand{\gc}{\cit{g}}
\newcommand{\hc}{\cit{h}}
\newcommand{\Zp}{{\mathbb{Z}_p}}

\newcommand{\Group}{\ensuremath{\mathbb{G}}\xspace}
\newcommand{\Hroup}{\ensuremath{\mathbb{H}}\xspace}
\newcommand{\map}{\mathbf{e}}

\newcommand{\prot}{\mathsf{Prot}}
\newcommand{\auxext}{\mathit{auxext}}
\newcommand{\auxsim}{\mathit{auxsim}}
\newcommand{\aux}{\mathit{aux}}
\newcommand{\state}{\mathit{state}}
\newcommand{\Alg}{\mathsf{Alg}}
\newcommand{\A}{\mathcal{A}}

\newcommand{\Sig}{\mathsf{Sig}}
\newcommand{\Gen}{\mathsf{Gen}}
\newcommand{\TK}{\mathsf{TK}}
\newcommand{\pseudokey}{{\sf TK}_{sim}}
\newcommand{\SK}{\mathsf{SK}}
\newcommand{\PK}{\mathsf{PK}}
\newcommand{\MSK}{\mathsf{MSK}}
\newcommand{\CT}{\mathsf{CT}}
\newcommand{\Screen}{\mathsf{Screen}}
\newcommand{\Setup}{\mathsf{Setup}}
\newcommand{\Keygen}{\mathsf{Keygen}}
\newcommand{\KeygenOut}{\mathsf{KeygenOut}}
\newcommand{\Transform}{\mathsf{Transform}}
\newcommand{\Decrypt}{\mathsf{Decrypt}}
\newcommand{\Decout}{\mathsf{DecOut}}
\newcommand{\compareequals}{\stackrel{?}{=}}
\newcommand{\numsigs}{\eta}

\title{A machine-generated proof of security for {\schemename}}
\author{}
\date{}
\maketitle

\section{$\KeygenOut$ Proof of Security}

Let $\listofmskvalues$ be the $\MSK$ variable(s) and $\listofrandomvalues$ be randomness selected in the $\Keygen$ algorithm. The $\Keygen$ algorithm runs to obtain the secret key, \\ $\SK = \{\secretkey\}$ and is computed as follows:

\begin{description}
\item {\sf SK}: \begin{multline*}  \originalkey \end{multline*}
\end{description}

\noindent
The $\KeygenOut$ algorithm selects blinding factors, $\listofbfs \in \Zp^*$. Let the new $\TK$ be computed from $\SK$ as follows:

\begin{description}
\item {\sf $\TK$}: \begin{multline*}  \transformkey \end{multline*}
\end{description}

\noindent
Note that a simulator given only public parameters ($\PK$) can formulate a simulated $\pseudokey$ by randomly selecting $\randomsecretkeys$ and computing: 

\begin{description}
\item {\sf $\pseudokey$}: \begin{multline*}  \pseudotransformkey \end{multline*}
\end{description}

Observe that this simulated $\pseudokey$ has an identical distribution to $\TK$.  Let $\keydefinitions$ be the new $\MSK$ variables and random values selected in $\KeygenOut$, and $\TK$ is computed as follows:

\begin{description}
\item {\sf $\TK$}: \begin{multline*}  \expandedtransformkey \end{multline*} 
\end{description}

% TODO: fine-tune
\section{$\Transform$ Proof of Correctness}

In the ${\schemename}$ scheme, we show the relevant portions to our transformation in the $\Decrypt$ algorithm. The ciphertext, $\CT = \{\ciphertext\}$ and the \\ $\SK = \{\secretkey\}$. 

\gutsofdecrypt

\noindent
[ NEED OFFICIAL TEXT HERE ] Next, we show how the computation of the original $\Decrypt$ algorithm is distributed between the $\Transform$ algorithm and the $\Decout$ algorithm.

\medskip \noindent
We first show the steps of the transform algorithm with respect to the transform key, $\TK$. 

\gutsoftransform

\noindent
We now show the final computation performed by the $\Decout$ algorithm which recovers the message and does not involve any pairing computations.

\gutsofdecout

\end{document}
